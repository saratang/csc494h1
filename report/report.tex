\documentclass[11pt]{article}

% These packages include nice commands from AMS-LaTeX
\usepackage{amssymb, amsmath, amsthm}
\usepackage{mathtools}

% Multiline comments
\usepackage{verbatim}

% Insert images
\usepackage{graphicx}

% For typing source code
\usepackage{listings}

% For captions
\usepackage{caption} \usepackage{subcaption}

% Allow for underlining.
\usepackage[normalem]{ulem}

% For CSV files
\usepackage{csvsimple}

% Margins
\usepackage[margin=1.5in]{geometry}

% From my yummy macros I decided not to import
\theoremstyle{definition} 
\newtheorem*{soln*}{Solution}
\newcommand{\vect}[1]{\overrightarrow{#1}} 
\newcommand{\Loss}{\mathcal{L}}
\newcommand{\mean}{\text{mean}} 
\newcommand{\var}{\text{var}}
\newcommand{\Mat}{\text{Mat}} 
\newtheorem*{theorem*}{Theorem}
\newcommand{\defeq}{\coloneqq}

\title{Fabricable Flows for Topology Optimization} \author{Sara Tang\\ 1001145058}

\begin{document}

\maketitle

\begin{abstract}
  We present a method of producing structually sound designs that can be
  assembled in a sequence of feasible steps. Traditionally, the topology
  optimization problem has been solved monolithically: however, in actuality
  structures are assembled not as a whole but in parts.

  In our method, we partition the given structure into a set of ``levels'',
  which are then optimized cumulatively to obtain a sequence of sound
  designs. We present an objective function that gives preference to beams in
  previous stages of the design to ensure a minimal set-difference between
  optimized levels. We demonstrate our method on a variety of different
  structures of varying degrees of detail and stability.
\end{abstract}

\section{Introduction}
The ability to construct structurally sound structures is essential to
(construction, models, engineering, I don't know Dave please halp). With the
advent of consumer-friendly 3D printers, we are creating more models than ever
(wow this sounds lame). But the size of such models are limited to the size of
the printer, and building larger designs brings about headaches of falling
structures and ohmygod I'll just write this later.

Talk about why it's important to build things in levels and not monolithically.

Talk about the different parts of the paper.

\section{Background}
Given a structure, presented as a set of vertices in 2- or 3-space and connecting
bars, and a set of forces acting on the vertices, the traditional topology
optimization aims to produce a superset of \textbf{sparse} and \textbf{possibly
overconstrained} bars that are \textbf{structurally stable}. 

At a high level, we do so by considering various deformations on our structure,
calculating the internal force on each bar given the deformation, and
attempting to balance the given external forces with the internal ones.

Consider the following structure in 2-space:

[INSERT IMAGE HERE LOL]

Let $u_i$ be a 2-vector representing the \textbf{displacement} at vertex $i$.
Let $\sigma_{ij}$ be a 2-vector representing the \textbf{stress} on some bar connecting
vertices $i$ and $j$. Each bar has an internal force dependent on the amount of
stress being applied on it, as well as an external force that is an input to the
problem. A \textbf{structurally stable} design is one such that internal and
external forces are balanced, that is, $f(\sigma) = f_{ext}$ for all bars.

\subsection{Measuring stress}
TODO: Consolidate when you use the vect sign and when you don't lol

Consider the following deformation of a single bar $(x_1, x_2)$:

[INSERT PICTURE HERE.]

where $u = (\vect{u_1}, \vect{u_2})$ (SHOULD BE COLUMN VEC) is infinitessimally
small in practice.

Define $A^T = [-I \; I]$ such that $\Delta u = vect{u_1} - \vect{u_2} = A^T u$.

Let $\vect{v}$ be the unit direction from $x_1$ to $x_2$. We will only consider
the deformation in direction $\vect{v}$, ie.

\[ \vect{v} = \frac{x_2 - x_1}{\lVert x_2 - x_1 \rVert} \]

As such, we project $\Delta u$ onto $\vect{v}$, then multiply by Young's
modulus to obtain the stress on $(x_1, x_2)$ given deformation $u$:

\begin{equation}
  \sigma \defeq E v^T \Delta u
\end{equation}

where $E$ is Young's modulus.

\subsection{Measuring internal force}
Given a bar with stress $\sigma$, the internal force is a two-vector $f = $
(ANOTHER COLUMN VECTOR), where

\begin{align*}
  f_1 &= \vect{n} \cdot \sigma \\
      &= \vect{n} E v^T \Delta u \\
      \intertext{Since $\vect{n}$ is exactly $v$,}
      &= v E v^T \Delta u \\
  f_2 &= - v E v^T \Delta u.
\end{align*}

Thus,

\begin{equation}
  f = \left[ \begin{matrix}
       I \\
      -I
  \end{matrix} \right]
  \left[ \begin{matrix}
      v E v^T \Delta u
  \end{matrix} \right]
\end{equation}

\subsection{All together now}
We now have a way of deriving the internal force on a single bar given its
deformation. We extend this notion to all bars in a given structure by defining
the matrix $K$ of dimensions $m \times n$ where $m$ is the number of bars and
$n$ is the number of vertices such that, given a displacement of vertices $u$,
$Ku$ returns a $m$-vector representing the internal force on each bar.

Furthermore, there is this $J$ matrix that I don't get, but basically $Ju = 0$,
where

\[ 
  J = \left[\begin{matrix}
    A^T v_1^T \\
    A^T v_2^T \\
    \vdots \\
    A^T v_n^T
\end{matrix} \right].
\]

Thus, our goal is to solve $u$ for the following constraints: $Ku = f$ and $Ju =
0$, that is

\begin{equation}
  \left[ \begin{matrix}
      K & J^T \\
      J & 0 \\
  \end{matrix} \right]
  \left[ \begin{matrix}
      U \\
      \lambda 
  \end{matrix} \right]
  =
  \left[ \begin{matrix}
      f \\
      0
  \end{matrix} \right]
  \label{eqn:primal}
\end{equation}

where $\lambda$ represents the tension on the bars.

Finally, we eliminate all bars where $\lambda = 0$, that is, bars that are no
subject to tension or compression.

\subsection{The Dual}

Although the above formulation will return to us a structurally sound design, it
will attempt to distribute the forces as evenly as possible, across all bars. We
wish to do the opposite: distribute the forces as sparsely as possible across
the bars, such that some bars will have no force on them and can thus be removed
from the structure. To do so, we consider the dual of the above formulation. By
using Gaussian elimination, we eliminate the $U$ from Equation \ref{eqn:primal} to
obtain:

\begin{equation}
  -(J K^{-1} J^T) \lambda = J^T K^{-1} f
\end{equation}

By representing $\lambda$ as a difference of tensions and compressions, $\lambda_T
- \lambda_C$, where $\lambda_T, \lambda_C \ge 0$, we can express sparsity as
minimizing the $L_1$-norm, that is: 

\[
  \lVert \lambda_T - \lambda_C \rVert = \sum \lambda_T + \sum \lambda_C
\]

Thus, we obtain the following linear optimization problem:

\begin{equation}
  \begin{split}
  & \lambda^*_T, \lambda^*_C = \arg \min \sum \lambda_T + \sum \lambda_C \\
  \text{subject to} & -(J K^{-1} J^T) (\lambda_T - \lambda_C) = J^T
  K^{-1} f \\
  & \lambda_T, \lambda_C \ge 0
  \end{split}
  \label{eqn:dual}
\end{equation}

which, in addition to the constraints expressed in Equation \ref{eqn:primal},
also attempts to sparsify the tensions and compressions across all bars.

\subsection{Overconstraining the problem}
As an additional note, we may augment the original design with additional bars
to ensure there is a structurally sound solution.

[PRESENT AN EXAMPLE.]

\section{Implementation}

\subsection{Augmentation 1: Levels}
Our first step into the unknown (lol) is to determine the substructure to be
optimized at each step in our sequence. A simple and intuitive way of doing this
is to label each bar with a level, from the ground up. This can be easily done
with a breadth-first search starting at the ground vertices.

\subsection{Augmentation 2: A better objective function}
We may not wish to weight each bar the same. For example, we may prefer bars in
the original structure over overconstrained ones. Thus, instead of simply
summing up the tensions and compression, we may assign a larger weight to
tensions and compressions corresponding to overconstrained bars and a smaller
(or even $0$) weight to tensions and compression corresponding to original ones.

\section{Results}

\section{Conclusion and Future Work}

\end{document}

